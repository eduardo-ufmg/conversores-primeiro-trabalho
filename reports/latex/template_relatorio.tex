\documentclass[12pt, a4paper]{article}

% Pacotes essenciais
\usepackage[utf8]{inputenc}
\usepackage[brazilian]{babel}
\usepackage[T1]{fontenc}
\usepackage{geometry}
\usepackage{graphicx}
\usepackage{amsmath}
\usepackage{amsfonts}
\usepackage{amssymb}
\usepackage{float}
\usepackage{subcaption}
\usepackage{booktabs}
\usepackage{siunitx}
\usepackage{hyperref}
\usepackage{listings}
\usepackage{xcolor}

% Configurações de página
\geometry{left=3cm, right=2cm, top=3cm, bottom=2cm}

% Configurações do siunitx
\sisetup{
    output-decimal-marker = {,},
    group-separator = {.},
    group-minimum-digits = 4
}

% Configurações para código MATLAB
\lstdefinestyle{matlab}{
    language=Matlab,
    basicstyle=\ttfamily\footnotesize,
    keywordstyle=\color{blue},
    commentstyle=\color{green!40!black},
    stringstyle=\color{red},
    numberstyle=\tiny\color{gray},
    numbers=left,
    frame=single,
    breaklines=true,
    captionpos=b
}

% Informações do documento
\title{Análise de Conversores usando FEMM, MATLAB e Simulink}
\author{Nome do Autor}
\date{\today}

\begin{document}

\maketitle

\tableofcontents
\newpage

\section{Introdução}

Este relatório apresenta a análise de conversores de potência utilizando a integração entre as ferramentas FEMM (Finite Element Method Magnetics), MATLAB e Simulink.

\subsection{Objetivos}

\begin{itemize}
    \item Modelar componentes magnéticos usando FEMM
    \item Extrair parâmetros magnéticos para simulação
    \item Simular comportamento do conversor em Simulink
    \item Analisar performance e eficiência
\end{itemize}

\subsection{Metodologia}

A metodologia adotada segue as seguintes etapas:

\begin{enumerate}
    \item Modelagem magnética em FEMM
    \item Extração automatizada de parâmetros via MATLAB
    \item Implementação do modelo de circuito em Simulink
    \item Análise dos resultados de simulação
\end{enumerate}

\section{Modelagem Magnética (FEMM)}

\subsection{Geometria do Componente}

Descreva aqui a geometria utilizada no modelo FEMM.

\begin{figure}[H]
    \centering
    \includegraphics[width=0.8\textwidth]{../figures/femm_geometry.png}
    \caption{Geometria do componente magnético modelado no FEMM}
    \label{fig:femm_geometry}
\end{figure}

\subsection{Propriedades dos Materiais}

\begin{table}[H]
    \centering
    \caption{Propriedades dos materiais utilizados}
    \label{tab:materiais}
    \begin{tabular}{@{}lcc@{}}
        \toprule
        Material & Permeabilidade Relativa & Condutividade (\si{\siemens\per\meter}) \\
        \midrule
        Núcleo Ferrite & 2000 & 0 \\
        Cobre & 1 & \num{5.8e7} \\
        Ar & 1 & 0 \\
        \bottomrule
    \end{tabular}
\end{table}

\subsection{Resultados da Análise Magnética}

Apresente aqui os resultados obtidos da análise FEMM.

\begin{figure}[H]
    \centering
    \begin{subfigure}{0.45\textwidth}
        \includegraphics[width=\textwidth]{../figures/flux_density.png}
        \caption{Densidade de fluxo magnético}
        \label{fig:flux_density}
    \end{subfigure}
    \hfill
    \begin{subfigure}{0.45\textwidth}
        \includegraphics[width=\textwidth]{../figures/field_lines.png}
        \caption{Linhas de campo magnético}
        \label{fig:field_lines}
    \end{subfigure}
    \caption{Resultados da análise magnética}
    \label{fig:magnetic_analysis}
\end{figure}

\section{Extração de Parâmetros (MATLAB)}

\subsection{Script de Integração}

O script desenvolvido para integração entre FEMM e MATLAB permite a extração automatizada dos parâmetros magnéticos:

\begin{lstlisting}[style=matlab, caption=Exemplo de extração de parâmetros]
% Carregar modelo FEMM
params = femm_matlab_integration('models/femm/inductor.fem', ...
                                'current', 2, 'frequency', 100);

% Exibir resultados
fprintf('Indutância: %.2f mH\n', params.inductance * 1000);
fprintf('Resistência: %.2f mΩ\n', params.resistance * 1000);
\end{lstlisting}

\subsection{Parâmetros Extraídos}

\begin{table}[H]
    \centering
    \caption{Parâmetros magnéticos extraídos}
    \label{tab:parametros}
    \begin{tabular}{@{}lcc@{}}
        \toprule
        Parâmetro & Valor & Unidade \\
        \midrule
        Indutância & \num{150} & \si{\micro\henry} \\
        Resistência & \num{25} & \si{\milli\ohm} \\
        Energia Magnética & \num{2.5} & \si{\milli\joule} \\
        Densidade de Fluxo Máx. & \num{0.8} & \si{\tesla} \\
        \bottomrule
    \end{tabular}
\end{table}

\section{Simulação do Conversor (Simulink)}

\subsection{Modelo do Circuito}

Descreva o modelo implementado em Simulink.

\begin{figure}[H]
    \centering
    \includegraphics[width=0.9\textwidth]{../figures/simulink_model.png}
    \caption{Modelo do conversor implementado em Simulink}
    \label{fig:simulink_model}
\end{figure}

\subsection{Resultados de Simulação}

\begin{figure}[H]
    \centering
    \begin{subfigure}{0.45\textwidth}
        \includegraphics[width=\textwidth]{../figures/voltage_waveform.png}
        \caption{Forma de onda da tensão}
        \label{fig:voltage}
    \end{subfigure}
    \hfill
    \begin{subfigure}{0.45\textwidth}
        \includegraphics[width=\textwidth]{../figures/current_waveform.png}
        \caption{Forma de onda da corrente}
        \label{fig:current}
    \end{subfigure}
    \caption{Formas de onda do conversor}
    \label{fig:waveforms}
\end{figure}

\section{Análise de Resultados}

\subsection{Eficiência do Conversor}

A eficiência do conversor foi calculada considerando as perdas nos componentes magnéticos:

\begin{equation}
    \eta = \frac{P_{out}}{P_{in}} = \frac{P_{out}}{P_{out} + P_{perdas}}
\end{equation}

\subsection{Comparação com Especificações}

Compare os resultados obtidos com as especificações de projeto.

\section{Conclusões}

Apresente as principais conclusões do trabalho.

\subsection{Trabalhos Futuros}

Sugira possíveis extensões e melhorias para o trabalho.

\bibliographystyle{ieeetr}
\bibliography{references}

\appendix

\section{Código MATLAB}

Inclua aqui códigos MATLAB relevantes que não foram apresentados no texto principal.

\section{Parâmetros de Simulação}

Liste os parâmetros utilizados nas simulações.

\end{document}